% Created 2021-11-05 pią 18:04
% Intended LaTeX compiler: pdflatex
\documentclass[11pt]{article}
\usepackage[utf8]{inputenc}
\usepackage[T1]{fontenc}
\usepackage{graphicx}
\usepackage{grffile}
\usepackage{longtable}
\usepackage{wrapfig}
\usepackage{rotating}
\usepackage[normalem]{ulem}
\usepackage{amsmath}
\usepackage{textcomp}
\usepackage{amssymb}
\usepackage{capt-of}
\usepackage{hyperref}
\usepackage{minted}
\author{Patryk Kaniewski}
\date{\today}
\title{}
\hypersetup{
 pdfauthor={Patryk Kaniewski},
 pdftitle={},
 pdfkeywords={},
 pdfsubject={},
 pdfcreator={Emacs 27.2 (Org mode 9.4.4)}, 
 pdflang={English}}
\begin{document}

\tableofcontents \clearpage\section{Firma kurierska paczki caly kraj}
\label{sec:orgfa62d01}
\begin{enumerate}
\item Okreslic jakie usługi i aplikacje będą wykorzystywane
\item Dokonać analizy zagrozen i ryzyk związanych z użytkownikiem aplikacji, usług i infrastruktury sieciowej
\item Sklasifikowac zagrozenia pod bezpiecznstwa ocenic realność i przechowywanych w systemie
\end{enumerate}

\section{Okreslic jakie usługi i aplikacje będą wykorzystywane}
\label{sec:orgba3a39a}
App1 odbierajacy musi otrzymwac powiadomienia (mail/aplikacja)
App2 Kurier musi mieć możliwość odebrania paczki z sortownii (skanowania jak wkłada do samochodu) musi mieć dostep do listy wszystkich paczek (adresy ITP.) 
App3 Sortownia rejestrować/wyrejestrowywać, musi mieć możliwość organizowania paczek do transportu
App4/API Przyjmowanie zlecen od klientów (np. sklepy internetowe)



App1 front end (mail/applikacja na telefon)
App2 front end (aplikacja dla kuriera) + potwierdzenie przyjęcia
App3 Back end
App4/API frontend (sklep) + backend (przyjmowanie)


\section{Dokonać analizy zagrozen i ryzyk związanych z użytkownikiem aplikacji, usług i infrastruktury sieciowej}
\label{sec:org7f43e7b}
App1 
    • zagrozenie tylko lokalne dla odbierającego, komunikacja jest jedno kierunkowa
App2
    • Dane osobowe odbierających, możliwość rejestrowania paczek jako dostarczone (kradzież paczek)
App3:
    • Pelna kontrola nad systemem (kradzież paczek, danych, tajemnic firmowych)
App4/API(front)
    • zagrozenie tylko lokalne dla zlecającego
App4 backend
    • zmiana danych przesylki, kradzież, dane osobowe

\section{Sklasifikowac zagrozenia pod bezpiecznstwa ocenic realność i przechowywanych w systemie}
\label{sec:org71ed77f}
(glowne wektory), zle programowanie, denial of service sa zawsze

App1 – zle programowanie
App2 – kradzież, slabe zabezpieczenia (fałszowanie potwierdzen)
App3 – czynnik ludzki (zabezpieczenia fizyczne) i wektory App3, App4/API (zla konfiguracja, podszywanie się)
App4 – man in the middle
\end{document}
